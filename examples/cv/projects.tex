\cvsection{Projects}
%------------------------------------------------------------------------------------
%	MSC FINAL PROJECT
%------------------------------------------------------------------------------------

%------------------------------------------------------------------------------------
%	MSC LITERATURE REVIEW
%------------------------------------------------------------------------------------
\begin{cventries}
	\cventry
		{M.Sc. Thesis}
		{Load balancing for adaptive radiation transport simulations}
		{London, UK}
		{Feb. 2018 - PRESENT}
		{
			\begin{cvitems}
%				\item {Distinguish between parabolic, hyperbolic and elliptic Partial Differential Equations (PDEs)}
				\item {Study and outline the use cases between FDM, FEM, FVM, DGFEM, $S_n$, $P_n$ and wavelets.}
				\item {Understand the functionality of different linear iterative solvers: Jacobi, Gauss-Seidel and Symmetric Successive Over Relaxation.}
				\item {Study the application of global and goal based adaptive methods on transport problems.}
				\item {Benchmark, improve and optimise existing code for load balancing of radiation transport problems.}
			\end{cvitems}
		}

%------------------------------------------------------------------------------------
%	BSC DISSERTATION
%------------------------------------------------------------------------------------
	\cventry
		{B.Sc. Dissertation}
		{Investigating the transition from Molecular Dynamics to Smoothed Particle Hydrodynamics}
		{Egham, UK}
		{Sept. 2016 - Apr. 2017}
		{
			\begin{cvitems}
				\item {Investigated the existence of a continuous transition between Molecular Dynamics (MD) and Smooth Particle Hydrodynamics (SPH) by creating computer simulations in C++ and performing data analysis in Python.}
				\item {Used an only repulsive pair potential to 		simulate a fluid with periodic boundary conditions.}
				\item {Parameters of the pair potential were altered resulting into a weakening of the interatomic forces and “softening” of the force distribution of the fluid.}
				\item {Using principals of statistical mechanics such as Radial Distribution Functions (RDF), Velocity Autocorrelation Functions (VAF) and Mean Square Displacement (MSD), quantitative observations were made for the transition limits between the two models (MD and SPH).}
				\item {It was deduced that there exists a continuous transition between MD and SPH for a small range of parameters of the pair potential. Weak pair potentials force the particles into clusters with infinitely small separation distances. For very weak potentials, the fluid could be accurately approximated by an ideal gas. The SPH limit could be approximated but never fully obtain a uniform distribution of forces between the particles, due to physical limitations of the potential (compared to Lucy or Monaghan potentials).}
			\end{cvitems}
		}	
	
%------------------------------------------------------------------------------------
%	PH 3040 NUCLEAR FISSION - REVIEW ARTICLE
%------------------------------------------------------------------------------------
%	\cventry
%	{Review Article for Fission -  PH3040~Energy}
%	{Reactor Designs in Nuclear Fission, Present-day and Future Alternatives}
%	{Egham, UK}
%	{Mar. 2017}	
%	{
%		\begin{cvitems}
%			\item {Analysed the design and performance of GEN-II and GEN-IV nuclear fission reactors.}
%			\item {Provided the theoretical background for the process of induced nuclear fission.}
%			\item {Reactor designs examined:\\}
%			\begin{cvitems}
%				\item [--] {Boiling Water Reactors (BWR).}
%				\item [--] {Pressurised Water Reactors (PWR).}
%				\item [--] {Liquid Metal (Fast Breeder) Reactors (LMFBR).}
%				\item [--] {Liquid Fluoride-Thorium Reactors (LFTR).\\}
%			\end{cvitems}
%		\end{cvitems}
%	}
	
%------------------------------------------------------------------------------------
%	PH 3040 - WENDELSTEIN 7-X - REVIEW ARICLE
%------------------------------------------------------------------------------------
	\cventry
	{Review Article for Fusion -  PH3040~Energy}
	{Nuclear Fusion: Stellarators and the Wendelstein 7-x Fusion Reactor}
	{Egham, UK}
	{Apr. 2017}
	{
		\begin{cvitems}
			\item {Provided a historical overview of the nuclear fusion reactor designs.}
			\item {Discussed the physical theory to achieve a nuclear fusion reaction and the conditions required to sustain it.}
			\item {Outlined the benefits of using stellarator designs, like W 7-X, to achieve continuous fusion.}
		\end{cvitems}
	}

%------------------------------------------------------------------------------------
%	CHAOS THEORY GROUP PROJECT
%------------------------------------------------------------------------------------
%	\cventry
%	{Group Project. Responsible: Report Writing, Presentation, Group Organisation}
%	{Chaos Theory and Strange Attractors}
%	{Egham, UK}
%	{Mar. 2015 - Apr. 2015}
%	{
%		\begin{cvitems}
%			\item {Stated the main principles of chaos theory and derived Lorenz’s differential equations.}
%			\item {Used Wolfram Mathematica to plot strange attractors using a range of different initial conditions and parameters.}
%			\item {Observations were made based on the results about the behaviour of chaotic systems, using Lyapunov’s exponent.}
%		\end{cvitems}
%	}
	
%------------------------------------------------------------------------------------
%	IB EXTENDED ESSAY - DAMPING COEFFICIENT PENDULUM
%------------------------------------------------------------------------------------
%------------------------------------------------------------------------------------
% GALLERY WITH... HOPE
%------------------------------------------------------------------------------------
\cventry
{Non-profit organisation}
{Gallery with... ELPIDA (Hope)}
{Athens, Greece}
{May. 2012}
{
	\begin{cvitems}
		\item {Organised, with two more individuals, a 3-day art gallery focused on charity, \newline with the aid of \textit{Piraeus Bank Group Cultural Foundation} and \textit{Association of Friends and Children with Cancer ``ELPIDA”}.}
		\item {Intent of the gallery, was to promote hope and give back to people who were in need during times of hardship.}
		\item {152 art pieces from the 1st workshop of Athens’s Art School, were displayed, in Moraitis School, raising a total of \euro 75,000.}
		\item {The profits were used for two purposes:\\
		\begin{cvitems}
			\item[--] {Support the noble cause of the \textit{Association of Friends and Children with Cancer ``ELPIDA”} and its president’s \newline Marianna~V.~Vardinoyannis.}
			\item[--] {Contribute in the combat of youth unemployment by employing and promoting young artists.}
		\end{cvitems}
				}
	\end{cvitems}
}
\end{cventries}